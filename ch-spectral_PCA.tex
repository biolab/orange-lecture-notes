\chapter{PCA on spectral data}
\label{ch:spectral-PCA}



\begin{wrapfigure}{o}{0.5\textwidth}
    \centering
    \vspace{-2cm}
    \includegraphics[width=0.55\textwidth]{graphics/ch-spectral_PCA/spectral_PCA-fig1.png}
\end{wrapfigure}

In this lesson we will explore the capabilities of \mutation\ for principal component analysis (PCA) on spectroscopy data. As usual, we will use the Liver Spectroscopy dataset. Connect \textit{Datasets} to the \textit{PCA} widget, choose the first 5 principal components and then connect \textit{PCA’s} default output, "Transformed Data", into the \textit{Scatter Plot}. 

We see that the first two principal components separate majority compounds in that part of the tissue well. 


\begin{figure}[h]
\hspace{-1cm}\stackinset{r}{-0.4\linewidth}{t}{+0.1\linewidth}
  {\includegraphics[scale=0.4]{graphics/ch-spectral_PCA/scatterplot.png}}
  {\includegraphics[scale=0.4]{graphics/ch-spectral_PCA/pca.png}}
  \caption{We chose not to normalize variables in \textit{PCA}. Why?}
  \label{fig:spectral-PCA-fig2}
\end{figure}

\begin{wrapfigure}{o}{0.8\textwidth}
  \centering
  \vspace{-0.5cm}
  \includegraphics[width=0.8\textwidth]{graphics/ch-spectral_PCA/spectral_PCA-fig3.png}%
  \caption{The curve under the cursor is highlighted. A tooltip will appear after some time. If clicked, the curve will be selected.}
  \label{fig:spectral-PCA-fig3}
\end{wrapfigure}

To see what different principal components represent, connect \textit{PCA’s} "Components" output (be careful, \textit{PCA} has 3 outputs) into \textit{Spectra}. Wondering which principal component is highlighted in the following screenshot? Wait for the tooltip...

Let's extend our workflow. In the next example, we are also using the “Data Subset” input. The \textit{Scatter Plot} and \textit{Spectra (2)} widgets on the right get both the whole data and a subset of it.

\begin{wrapfigure}{o}{0.7\textwidth}
  \centering
%   \vspace{-2cm}
  \includegraphics[width=75mm]{graphics/ch-spectral_PCA/spectral_PCA-fig4.png}%
  \label{fig:spectral-PCA-fig4}
\end{wrapfigure}

If we connect the \textit{PCA} (Transformed data output) to \textit{Spectra (1)}, we see each transformed spectrum on a line plot. As we can see, some classes have outliers. 

To find out more, select one of them: move your mouse cursor to a curve -—it will be highlighted—- and click it. The selected curve changes to a dotted line and is sent to the output. Then, connect that output to a \textit{Scatter Plot} and another \textit{Spectra} widget (both to its Data Subset inputs). We can then see the spectrum in the original space (\textit{Spectra (2)} widget) and the space of principal components (\textit{Scatter Plot}).

\begin{figure*}[h]
%   \centering
  \vspace{3cm}\stackinset{r}{0pt}{t}{+0.35\textwidth}
  {\includegraphics[width=0.48\textwidth]{graphics/ch-spectral_PCA/spectral_PCA-fig6.png}}
  {\includegraphics[width=\textwidth]{graphics/ch-spectral_PCA/spectral_PCA-fig5.png}}
  \vspace{-3.5cm}
  \caption{The selected spectrum's curve on the left is drawn with a dashed line and the corresponding original spectrum is highlighted on the right. \textit{Scatter Plot} shows the position of the selected spectrum in the PCA space.}
  \label{fig:spectral-PCA-fig5}
\end{figure*}
