\chapter*{Introduction}


\begin{figure*}[t!]
  \includegraphics[width=\linewidth]{graphics/\mutation-intro-fig.jpg}%
  \label{chfig:intro}%
\end{figure*}


\newthought{This book} showcases machine learning and scientific data analysis problems using \mutation\cite{\mutation} through easily reproducible workflows.

First, let us acknowledge that \mutation\ is but a pre-packaged version of Orange\cite{orange} and the Orange Spectroscopy add-on\cite{git\mutation}.

The examples presented here introduce the features of \mutation\ through common data analysis tasks. You will see how common data mining can be accomplished through visual programming. We will also apply the same techniques to spectral data and hyperspectral images. \marginnote{\newthought{These notes include} \mutation\ workflows and visualizations we will construct during the course. 

The original notes were written by the members of the SMIS beam line of the SOLEIL Synchrotron, and the Bioinformatics Lab at University of Ljubljana and are extensions of the notes by Blaž Zupan, Janez Demšar and Marko Toplak.
}
If you are already familiar with data analysis, the methodological aspects of this course will seem simple, but you will have more time to absorb \mutation\ and the \mutation\ philosophy — try to think of what is happening behind the scenes.


\begin{figure*}[b!]
    % \centering
    \includegraphics[width=30mm]{CC-BY-SA_icon_white.png}
    \label{fig:CC-BY-SA_icon}
\end{figure*}
