\chapter{Analiza sentimenta}
\label{ch:analiza-sentimenta}

Kaj pa kakšni novi podatki? Na primer z omrežja Twitter? V gradniku Corpus kliknite na spustni meni in izberite election-2016-tweets.tab. Ta korpus vsebuje 6000 tvitov Hillary Clinton in Donalda Trumpa v času predvolilne kampanje leta 2016.

Najprej moramo podatke predobdelati. Uporabili bomo poseben razčlenjevalnik enot (tokenizer) Tweet, ki je bil naučen na miljonih tvitov.\marginnote{Tviti pogosto zahtevajo posebno predprocesiranje. Najprej odstranite vse povezave (remove url), ker so običajno nezanimive za analizo. Nato uporabite prednaučeni Tweet razčlenjevalnik, ki obdrži enote kot so ‘@omemba’, ‘\#oznaka’ and emotikoni :). Za konec uporabite še filter Regexp, ki odstrani ločila.}

Tokrat bomo imeli kar veliko unikatnih besed in stvari lahko postanejo precej počasne. Število besed lahko zmanjšamo tako, da uporabimo tehniko Document frequency. Nižjo mejo bomo nastavili na 10 - tako bomo obdržali vse besede, ki se pojavijo v več kot 10 tvitih. Super, sedaj imamo znosnih 1000 enot.

Tokrat si bomo podatke pogledali z vidika čustev, ki jih tviti izražajo. Dodajte gradnik Sentiment Analysis. Ta deluje tako, da besede iz dokumentov primerja s slovarjem pozitivnih in negativnih besed in na koncu izračuna oceno, kakšno čustvo zaznamuje tvit - pozitivno, negativno ali nevtralno.

Preden pa pogledamo rezultat, bomo morali še nekaj spremenljivk umakniti iz pogleda (glejte sliko). Uporabite gradnik Select Columns in premaknite vse dodatne informacije o tvitih v polje meta attributes.

\begin{figure}[h]
    \centering
    \includegraphics[width=\linewidth]{sentiment-1.png}%
    \caption{Nastavitve: 
    
    Razčlenjevalnik Tweet se uporablja za tvite, prednastavljeni razčlenjevalnik Regexp pa za običajno besedilo.
    
    Nastavite pogostost v dokumentih na 10. To bo odstranilo enote, ki se pojavijo v manj kot 10 tvitih.}
    \label{fig:011-sentiment}
\end{figure}
  
Nato dodajmo gradnik Heat Map, ki prikazuje številske vrednosti stolpcev z barvno shemo. Modra polja imajo nizko vrednost, rumena pa visoko. Z drugimi besedami, kjer je vrednost spremenljivke compound morda, je tvit negativen, kjer pa je rumena, je pozitiven.

\begin{figure*}[h]
    \includegraphics[width=\linewidth]{sentiment-workflow.png}%
    \caption{}
    \label{fig:011-sentiment-workflow}
\end{figure*}

V prikazu lahko označimo negativne tvite in jih pogledamo v Corpus Viewerju ali pa pogledamo, kdo jih je napisal z gradnikom Box Plot.

\begin{figure*}[h]
    \includegraphics[width=\linewidth]{sentiment-2.png}%
    \caption{}
    \label{fig:011-sentiment2}
\end{figure*}